\section{Methods}
\label{sec:methods}

For our method to analyze Tor browser artifacts, we created a setup
with a virtual machine and installed Ubuntu 18.04. Next, we installed
the Tor Browser and browsed to a website. Then we created a memory
image using LiME 1.8.1
\footnote{\url{https://github.com/504ensicslabs/lime}}. The Tor
browser virtual address space was afterwards extracted using
Volatility 2.6.1
\footnote{\url{https://www.volatilityfoundation.org/}}. This allowed
us to acquire dumps containing Tor artifacts.

%analyse documentation
Firefox was extensively analyzed during this research. Firefox is
written in multiple programming languages such as Python, C++, Rust
and JavaScript. However, C++ powers most of the mission critical code
with parts now being replaced with Rust. Firefox is also a large
software project, with over 36 million lines of code
\footnote{\url{https://www.openhub.net/p/firefox/analyses/latest/languages_summary}}.
Most of the documentation relating to low-level Firefox functionality
is scattered, not entry level friendly, and for JavaScript/web
development. Therefore it was difficult understanding the complicated
data structures and classes that make up Firefox, especially since most
of them use inheritance, polymorphism, and nested classes. Mozilla
also completed project Electrolysis in 2018
\footnote{\url{https://developer.mozilla.org/en-US/docs/Mozilla/Firefox/Multiprocess_Firefox}}.
This means that Firefox is now a fully multi-process web browser with
inter-process communication. This allows for more security and
sand-boxing, but makes analyzing the program harder. It consists of one
process for GUI and a separate process per tab.

%reverse engineering setup
For reverse engineering and analyzing the browser we used GDB
8.2.1-2. To gain a better understanding of what object we inspect, we
downloaded Tor Browser 8.0.6 with debugging symbols. This provides
additional information for executables.


