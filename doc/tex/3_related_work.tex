\section{Related work}
\label{sec:related_work}

There has been some previous work on analyzing web browser
artifacts. One research proved that it is possible to reconstruct
video content played in the Chrome browser by using locally buffered
cache data on the hard drive. \cite{horsman2018reconstructing}

Horsman also tried to prove if somebody viewed an image bases on the
hard drive cache files of both Mozilla Firefox and Google
Chrome. However, testing highlighted inconsistent cache behavior,
meaning that it may not possible to make an accurate assumption of
which images were viewed by a user based on cached content alone.
\cite{horsman2018didn}

Another interesting research was done on artifacts left behind by
private browser modes. All large web browsers were researched,
e.g. Chrome, Firefox and Safari. The majority of important artifacts
with regard to privacy are removed by the browsers, but users did not
take into account the impact of artifacts with low or medium valuation
being left behind \cite{tsalis2017exploring}. One of mitigation's
presented in this paper is storing data into the RAM. This decreases
the chance of confidential data leaks. However, as we present in our
paper, forensic experts can still extract artifacts from RAM,
disproving this mitigation. Also, these three papers use cache located
on the hard drive for finding artifacts as opposed to using the
memory.

A paper which resembles our technique, i.e. recovering artifacts from
RAM, is the research done by Yang et al, where they managed to recover
login credentials from private browsing sessions from RAM
\cite{yang2017applying}. This paper is interesting from a technical
perspective, but not so much from a forensic perspective.

The most important related work is the work done by a previous
Cybercrime and Forensics group in 2018. They managed to reconstruct
the entire GUI of a Chromium-based browser from a memory image
\cite{wikner2018reconstruct}. This work is very interesting, but
unfortunately not suited for this project because it is focused on a
different browser with different data structures and because it
utilized previous work for reversing these data structures in memory.
